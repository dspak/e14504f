\documentclass[11pt,]{article}
\usepackage[left=1in,top=1in,right=1in,bottom=1in]{geometry}
\newcommand*{\authorfont}{\fontfamily{phv}\selectfont}
\usepackage[]{mathpazo}


  \usepackage[T1]{fontenc}
  \usepackage[utf8]{inputenc}



\usepackage{abstract}
\renewcommand{\abstractname}{}    % clear the title
\renewcommand{\absnamepos}{empty} % originally center

\renewenvironment{abstract}
 {{%
    \setlength{\leftmargin}{0mm}
    \setlength{\rightmargin}{\leftmargin}%
  }%
  \relax}
 {\endlist}

\makeatletter
\def\@maketitle{%
  \newpage
%  \null
%  \vskip 2em%
%  \begin{center}%
  \let \footnote \thanks
    {\fontsize{18}{20}\selectfont\raggedright  \setlength{\parindent}{0pt} \@title \par}%
}
%\fi
\makeatother




\setcounter{secnumdepth}{0}



\title{Using microbial composition within sputum transcriptome data to stratify
patients by asthma severity  }



\author{\Large Nneoma Adaku\vspace{0.05in} \newline\normalsize\emph{}   \and \Large Daniel J Spakowicz\vspace{0.05in} \newline\normalsize\emph{Program in Computational Biology and Bioinformatics, Yale University,
New Haven, CT}   \and \Large Scott Strobel\vspace{0.05in} \newline\normalsize\emph{Department of Molecular Biophysics and Biochemistry; Yale University,
New Haven, CT}   \and \Large Faye Rogers\vspace{0.05in} \newline\normalsize\emph{}  }

\providecommand{\tightlist}{%
  \setlength{\itemsep}{0pt}\setlength{\parskip}{0pt}}

\date{}

\usepackage{titlesec}

\titleformat*{\section}{\normalsize\bfseries}
\titleformat*{\subsection}{\normalsize\itshape}
\titleformat*{\subsubsection}{\normalsize\itshape}
\titleformat*{\paragraph}{\normalsize\itshape}
\titleformat*{\subparagraph}{\normalsize\itshape}


\usepackage{natbib}
\bibliographystyle{plainnat}



\newtheorem{hypothesis}{Hypothesis}
\usepackage{setspace}

\makeatletter
\@ifpackageloaded{hyperref}{}{%
\ifxetex
  \usepackage[setpagesize=false, % page size defined by xetex
              unicode=false, % unicode breaks when used with xetex
              xetex]{hyperref}
\else
  \usepackage[unicode=true]{hyperref}
\fi
}
\@ifpackageloaded{color}{
    \PassOptionsToPackage{usenames,dvipsnames}{color}
}{%
    \usepackage[usenames,dvipsnames]{color}
}
\makeatother
\hypersetup{breaklinks=true,
            bookmarks=true,
            pdfauthor={Nneoma Adaku () and Daniel J Spakowicz (Program in Computational Biology and Bioinformatics, Yale University,
New Haven, CT) and Scott Strobel (Department of Molecular Biophysics and Biochemistry; Yale University,
New Haven, CT) and Faye Rogers ()},
             pdfkeywords = {asthma, microbiome, RNAseq, metatranscriptomics},  
            pdftitle={Using microbial composition within sputum transcriptome data to stratify
patients by asthma severity},
            colorlinks=true,
            citecolor=blue,
            urlcolor=blue,
            linkcolor=magenta,
            pdfborder={0 0 0}}
\urlstyle{same}  % don't use monospace font for urls



\begin{document}
	
% \pagenumbering{arabic}% resets `page` counter to 1 
%
% \maketitle

{% \usefont{T1}{pnc}{m}{n}
\setlength{\parindent}{0pt}
\thispagestyle{plain}
{\fontsize{18}{20}\selectfont\raggedright 
\maketitle  % title \par  

}

{
   \vskip 13.5pt\relax \normalsize\fontsize{11}{12} 
\textbf{\authorfont Nneoma Adaku} \hskip 15pt \emph{\small }   \par \textbf{\authorfont Daniel J Spakowicz} \hskip 15pt \emph{\small Program in Computational Biology and Bioinformatics, Yale University,
New Haven, CT}   \par \textbf{\authorfont Scott Strobel} \hskip 15pt \emph{\small Department of Molecular Biophysics and Biochemistry; Yale University,
New Haven, CT}   \par \textbf{\authorfont Faye Rogers} \hskip 15pt \emph{\small }   

}

}





\vskip 6.5pt

\noindent \doublespacing \section{Introduction}\label{introduction}

This is the script used to build a tree for Nneoma's fungus and assign
its taxonomy.

I started by rolling through the databases from the recent review
\url{http://jcm.asm.org/content/55/4/1011.full} to check if any would be
useful for this project.

\begin{itemize}
\tightlist
\item
  BOLD systems
  \url{http://v4.boldsystems.org/index.php/IDS_OpenIdEngine} only have
  ITS identification.
\item
  Looks like this is a good place for morphological features
  \url{https://aftol.umn.edu/} and can even make a nexus file to include
  in the tree -- but aftol has been lost? (goes to godaddy
  website\ldots{})
\item
  BROAD doesn't have an identification portal
\item
  EZBioCloud doesn't have a fungal id section
\item
  FungiDB is just genomics
\item
  UNITE is just ITS
\item
  IndexFungorum doesn't have an id search (but could be useful for
  morphology)
\item
  CBS can be searched directly for LSU and there are lots of good hits.
  However, I'd rather find a paper that has gone through the effort of
  identifying isolates with multiple loci
\item
  SILVA has an LSU search \url{https://www.arb-silva.de/}
\item
  Identity: 43.61, LCA tax SILVA: None
\item
  SSU Iden: 99.37, LCA tax. SILVA: None
\item
  RDP \url{http://rdp.cme.msu.edu/classifier/}
\item
  E14504F-LSU Root(100\%) Fungi(100\%) Basidiomycota(100\%)
  Agaricomycetes(100\%) Cantharellales(100\%) Ceratobasidiaceae(100\%)
  Thanatephorus(100\%)
\end{itemize}

The RDP result is strong, with 100\% confidence in the genus
Thanatephorus. The CBS searches also found organisms of either
Thanatephorus (telomorph) or Rhizoctonia (anamorph). This will very
likely be the genus to which E14504F belongs. In addition, Nneoma and I
found a few papers that deal with isolates of Rhizoctonia/Thanatephorus:

\begin{itemize}
\tightlist
\item
  \citep{gonzalez_ribosomal_2001} has a bunch of Thanatephorus isolates
  with genbank accession numbers for ITS and 28S, but nothing outside
  the genus (which is necessary to demonstrate the circumscription in
  this case).
\item
  \citep{tupac_otero_diversity_2002} just have ITS and have several
  genera that were isolated from orchids. It's more orchid-centric than
  fungus-centric.
\item
  \citep{lopez-chavez_proteomic_2016} defines a Thanatephorus isolate
  using ITS alone. The tree shows weak node support separating
  Thanatephorus from Ceratobasidium, but clearly their isolate is
  closest to a Thana.
\item
  \citep{gonzalez_phylogenetic_2016} does a really nice job of creating
  a multi-locus tree. This should be the model going forward.
\end{itemize}

\section{Methods}\label{methods}

I converted the table of genbank accession numbers from
\citep{gonzalez_phylogenetic_2016} to a google spreadsheet.

Here are the files that Nneoma created using the Staden package (pregap
\& gap). As soon as these have genbank accession numbers I'll add them
to the table so that they can be pulled with the other sequences from
the table.

\newpage
\singlespacing 
\renewcommand\refname{Results and Discussion}
\bibliography{Nneoma.bib}

\end{document}
